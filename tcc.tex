\documentclass[12pt]{article}

\usepackage{setspace}
\usepackage{amsmath,amssymb}
\usepackage{amsfonts}
\usepackage{graphicx}
\usepackage[pdftex,bookmarks=true,bookmarksopen=false,bookmarksnumbered=true,colorlinks=true,linkcolor=black]{hyperref}
\usepackage[utf8]{inputenc}
\usepackage{float}
\usepackage{pdfpages}

\usepackage[brazil]{babel}
%\usepackage{pstricks}%, egameps}

%\setlength{\textwidth}{17.2cm}
% \setlength{\textheight}{23cm}
%\addtolength{\oddsidemargin}{-22mm} 
%\addtolength{\topmargin}{-15mm} \addtolength{\evensidemargin}{-15mm}
%\setlength{\parskip}{1mm}
%\setlength{\baselineskip}{500mm}

\newtheorem{theorem}{Theorem}[section]
\newtheorem{assumption}{Assumption}
\newtheorem{acknowledgment}{Acknowledgment}
\newtheorem{algorithm}{Algorithm}
\newtheorem{axiom}{Axiom}
\newtheorem{case}{Case}
\newtheorem{claim}{Claim}
\newtheorem{conclusion}{Conclusion} 
\newtheorem{condition}{Condition}
\newtheorem{conjecture}{Conjecture}
\newtheorem{corollary}{Corollary}[section]
\newtheorem{criterion}{Criterion}
\newtheorem{defn}{Definition}[section]

\newtheorem{example}{Example}[section]
\newtheorem{exercise}{Exercise}
\newtheorem{lemma}{Lemma}[section]
\newtheorem{notation}{Notation}
\newtheorem{problem}{Problem}
\newtheorem{proposition}{Proposition}[section]
\newtheorem{remark}{Remark}
\newtheorem{solution}{Solution}
\newtheorem{summary}{Summary}
\newenvironment{proof}[1][Proof]{\textbf{#1.} }{\rule{0.5em}{0.5em}}

\begin{document}

\begin{titlepage}
\begin{center}
\textbf{\LARGE Fundação Getulio Vargas}\\ 
\textbf{\LARGE Escola de Matemática Aplicada}\\
\textbf{\LARGE Curso de Graduação em Matemática Aplicada}

\par
\vspace{170pt}
\textbf{\Large Título da dissertação}\\
\vspace{80pt}
\textbf{\Large Emanuel Bissiatti de Almeida}\\
\end{center}

\par
\vfill
\begin{center}
{{\normalsize Rio de Janeiro - Brasil}\\
{\normalsize \the\year}}
\end{center}
\end{titlepage}

\thispagestyle{empty}

\newpage
\begin{center}
\textbf{\LARGE Fundação Getulio Vargas}\\ 
\textbf{\LARGE Escola de Matemática Aplicada}\\
\textbf{\LARGE Curso de Graduação em Matemática Aplicada}

\par
\vspace{100pt}
\textbf{\Large Título da dissertação}


\par
\vspace{65pt}
``\textbf{Declaro ser o único autor do presente projeto de
monografia que refere-se ao plano de trabalho a ser executado para continuidade da monografia e ressalto que não recorri a qualquer forma de colaboração ou auxílio de terceiros para realizá-lo a não ser nos casos e para os fins autorizados pelo professor orientador.}''
\end{center}

\par
\vspace{65pt}
\begin{center}


\hrulefill

\vspace{5pt}
\textbf{\Large Nome}
\end{center}

\par
\vfill
\begin{center}
{{\normalsize Rio de Janeiro - Brasil}\\
{\normalsize \the\year}}
\end{center}

\thispagestyle{empty}

\newpage
\begin{center}
\textbf{\LARGE Fundação Getulio Vargas}\\ 
\textbf{\LARGE Escola de Matemática Aplicada}\\
\textbf{\LARGE Curso de Graduação em Matemática Aplicada}

\par
\vspace{100pt}
\textbf{\Large Título da dissertação}


\par
\vspace{65pt}

``\textbf{Projeto de Monografia apresentado à Escola de Matemática Aplicada como requisito parcial para continuidade ao trabalho de monografia.}''
\end{center}

\par
\vspace{65pt}
\begin{center}

\textbf{Aprovado em } \makebox[30pt]{\hrulefill}\textbf{ de }\makebox[120pt]{\hrulefill}\textbf{ de }\makebox[50pt]{\hrulefill}
\\
\vspace{5pt}
\textbf{Grau atribuído ao Projeto de Monografia:} \makebox[30pt]{\hrulefill}\\
\end{center}


\par
\vspace{40pt}
\begin{center}

\hrulefill

\vspace{5pt}
\textbf{Professor Orientador: }\\
\textbf{Escola de Matemática Aplicada}\\
\textbf{Fundação Getúlio Vargas}
\end{center}

\thispagestyle{empty}


\newpage
\tableofcontents
\thispagestyle{empty}

\newpage
\section{Introdução}

introdução...

\newpage
\section{Objetivo Final}

\subsection{xx}

referencial teórico... \footnote{Ver xx}.

\subsubsection{yy}

\newpage
\subsection{xx}

xx

\newpage
\section{Metodologia}

metodologia...

\subsection{xx}

xx

\newpage
\begin{figure}[H]
\centering
%left bottom right top
%\includegraphics[trim=10mm 00mm 00mm 00mm, scale=0.40]{tabela.pdf}
\begin{center}
Figura 1: tabela.
\end{center}
\end{figure}

\begin{figure}[H]
\centering
%left bottom right top
%\includegraphics[trim=10mm 00mm 00mm 00mm, scale=0.40]{tabela.pdf}
\begin{center}
Figura 2: tabela.
\end{center}
\end{figure}

\newpage
\subsubsection{yy}

yy

% \[\left\{ 
\begin{align*}
f_{i}(x) = (10x + 100), \qquad \text{(1)} \tag{1}\\
f_{ii}(x) = (20x + 200), \qquad \text{(2)} \tag{2} \\
f_{iii}(x) = (30x + 300), \qquad \text{(3)} \tag{3} \\
\end{align*} %\right.\]

xx

\begin{align*}
Vm_{i}(p,l) = ((-1.9141)p + 49.466)l + ((199.51)p - 10795.0), \text {$l$=0} \tag{4}
\end{align*}

\begin{align*}
f_{n}(y) = \frac{y}{1000}, \tag{5}
\end{align*}

\subsection{xx}

xx

%\[\left\{
\begin{align*} 
Funcao_{i}(p) = \gamma + \delta p + \theta p^2 + \omega p^3, \qquad \qquad \qquad \qquad \qquad \qquad \qquad \qquad \text{(6)} \tag{6} \\
\end{align*} %\right.\]

\newpage
\section{Resultados Esperados}

Nesta seção serão apresentados os resultados esperados...

\subsection{xx}

xx

\newpage
\section{Referências}

[1] a.

\noindent [2] b.

\noindent [3] c.

\noindent [4] d.


\end{document}